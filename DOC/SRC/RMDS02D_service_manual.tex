\documentclass[12pt,a4paper,oneside]{article}
\usepackage[colorlinks=true]{hyperref}
\usepackage[utf8]{inputenc}
\usepackage[czech]{babel}
\usepackage{graphicx}
\usepackage{pdfpages}
\textwidth 16cm \textheight 25cm
\topmargin -1.3cm 
\oddsidemargin 0cm
\pagestyle{empty}
\begin{document}
\title{Radiová meteorická detekční stanice RMDS02D}
\author{Jakub Kákona, kaklik@mlab.cz }
\maketitle

\begin{abstract}
Servisní manuál k rádiové detekční stanici RMDS02D sítě Bolidozor.
\end{abstract}

\begin{figure} [htbp]
\begin{center}
\includegraphics [width=120mm] {./img/RMDS02D_showcase1.JPG} 
\end{center}
\end{figure}

\begin{figure} [b]
\includegraphics [width=25mm] {./img/Bolidozor_web_QR.png} 
\end{figure}

\newpage
\tableofcontents

\section{Základní technické parametry}
\begin{table}[htbp]
\begin{center}
\begin{tabular}{|c|c|p{5cm}|}
\hline
\multicolumn{1}{|c|}{Parametr} & \multicolumn{1}{|c|}{Hodnota} & \multicolumn{1}{|c|}{Poznámka} \\ \hline
Napájení analogových obvodů & $\pm$12V &  cca 35mA \\ \hline
Napájení digitálních obvodů & 2x +5V &  2A \\ \hline
Napájení předzesilovače  & 9-12V &  500 mA \footnote{Chráněno vratnou PTC pojistkou 750mA na desce přijímače} \\ \hline
Pracovní Frekvenční rozsah  & 0,5 - 200 MHz & Obvykle 143.05 MHz $\pm$ 192 kHz \\ \hline
Vzorkování I/Q signálu  & 16bit@96kHz & \\ \hline
Celkový zisk & 60-90 dB & Volitelně podle konfigurace LNA \footnote{Lze ovlivnit jumperem na desce přijímače}\\ \hline
Celkové šumové číslo & $<$ 10 dB & \\ \hline
Minimální detekovatelný signál & -135dBm  &  Při konfiguraci v síti Bolidozor.\\ \hline
Přesnost časové synchronizace &  65ns (1$\sigma$) &  Používán signál GPS L1.\\ \hline
Rozlišení kontrolního měření frekvence LO &  0.1 Hz &  Přesnost je závislá na kvalitě signálu GPS.\\ \hline
\end{tabular}
\end{center}
\end{table}

\newpage

\section{Konstrukce detekční stanice}

\begin{figure}[htbp]
\begin{center}
\includegraphics [width=120mm] {../../SCH/RMDS02D_system.png} 
\end{center}
\caption{Konstrukce antény používané na detekční stanici.}
\end{figure}


\subsection{Přijímací obvody}

Rádiový


\subsection{SDR přijímač}

V přijímací stanici je využíván přijímač MLAB SDRX01B \cite{SDRX01B_pdf}. Jde o SDR přijímač zkonstruovaný původně pro účely amatérské radioastronomie a je proto optimalizovaný pro realizaci škálovatelných rádiových systémů. Přijímač vyžaduje externí lokální oscilátor, který je v případě této konstrukce realizovaný zapojením modulu CLKGEN01B s obvodem Si570, který je řízen přes sběrnici I2C.  Frekvence tohoto lokálního oscilátoru je měřena 3x za minutu a logována do souboru. Po změření se navíc frekvence automaticky koriguje na nastavenou hodnotu. Fázový šum lokálního oscilátoru je maximálně –105~dBc/Hz (Při offsetu 100~Hz)

\begin{figure}[htbp]
\begin{center}
\includegraphics [width=120mm] {./img/SDRX01B_Top_Big.JPG} 
\end{center}
\caption{Hlavní modul přijímače SDRX01B}
\end{figure}

Díky měření frekvence lokálního oscilátoru je ve stanici v podstatě obsažena GPSDO kmitočtová reference, která napájí přijímač. Výstupem přijímače je tak komplexní I/Q signál, který je v této verzi stanice digitalizován externí USB zvukovou kartou Behringer UCA 202 U-CONTROL. Tento audio kodek vzorkuje 48kHz s rozlišením 16bit ve dvou kanálech.  Zvuková karta bude v následující verzi stanice nahrazena digitalizační jednotkou SDR-Widget. Která už může mít vzorkovací hodinový signál též odvozen od hlavního lokálního oscilátoru. 

\subsection{Časová synchronizace}

Časová synchronizace je velmi dúležitá ve vědeckých měřeních. Konstrukce využívá dvě metody časové synchronizace NTP a signál z GPS.  NTP je využíváno pro časovou synchronizaci systémových hodin staničního počítače a používá se i pro generování názvů výstupních datových souborů. Přesnost tohoto synchronizačního kanálu závisí na kvalitě internetového připojení, resp době pingu k NTP serveru.  Proto je na stanici zaveden ještě zdroj časových značek z GPS, který umožňuje značkovat výstupní data s rozlišením až na vzorkovací frekvenci digitalizační jednotky. 
Jako GPS přijímač je ve stanici využíván MLAB modul GPS01A s čipsetem uBlox-LEA6. Použitá GPS anténa je keramická patch s integrovaným předzesilovačem.

\subsubsection{Měření frekvence lokálního oscilátoru}

Stanice je vybavena automatizovaným systémem kontroly lokálního oscilátoru. Zařízení umožňuje měření frekvence lokálního oscilátoru oproti dlouhodobému časovému normálu z GPS. Měření se provádí tak, že signál z lokálního oscilátoru generovaného modulem CLKGEN01B se vydělí modulem CLKDIV01A a vydělený výstupní signál je veden do čítače v PIC16F887. Kde jsou na hrubo počítány tako vydělené hodinové pulzy. 
Modul GPS01A je pak nastaven tak, aby generoval 100us dlouhý impuls každých celých 10s UT.  Tento impuls je pak využit ke zahradlování děličky CLKDIV01A.  Zároveň impuls vede na vstup přerušení PIC16F887 a ten pak provede vyčtení čítače a interních stavů děličky tím, že projde všechny možné dělící poměry a přečte výstupní stav děličky. Výsledkem je absolutně změřený počet pulzů za 10s.  Toto číslo je následně uloženo v registrech mikrokontroleru PIC16F887. Vyčtení těchto registrů proběhne ze staničního počítače přes sběrnici I2C každou 5. 25. a 45. sekundu UT času. \footnote{Měření je takto rozloženo díky tomu, že obsluha I2C probíhá také pod přerušením a při vyčítání může dojít k poškození měřené hodnoty vynecháním přerušení obsluhy čítače.}
Protože při zahradlování děličky dojde k odpojení lokálního oscilátoru od směšovače v SDRX01B, tak na výstupním signálu se objeví impuls s širokým frekvenčním spektrem, který slouží jako časová značka ve výstupních datech.  Časová synchronizace tak zároveň přesně značkuje výstupní data hodinami odvozenými z GPS signálu. Tvar těchto časových značek je navíc odvozen od impulzní odezvy přijímacího řetězce, takže na něm lze pozorovat případné nežádoucí změny parametrů. 

\subsection{Staniční počítač}

Jako staniční počítač je využíván Hardkernel ODROID-U3, což je čtyřjádrový počítač postavený na architektuře ARM. K tomuto počítači je ještě obvykle připojen USB flash disk, který slouží jako krátkodobé uložiště pro zaznamenaná data.  Technické parametry použitého ARM počítače jsou následující: 

\begin{itemize}
\item 1.7GHz Quad-Core processor and 2GByte RAM
\item 10/100Mbps Ethernet with RJ-45 LAN Jack
\item 3 x High speed USB2.0 Host ports
\item Audio codec with headphone jack on board
\item GPIO/UART/I2C ports
\end{itemize}

Komponenty stanice jsou k počítači připojeny přes rozhraní USB: (Zvuková karta, GPS přijímač) a přes rozhraní I2C: (Lokální oscilátor, měření frekvence)

\begin{figure}[htbp]
\begin{center}
\includegraphics [width=140mm] {./img/u3rev05boarddetail.jpg} 
\end{center}
\caption{Deska počítače Hardkernel ODROID-U3}
\end{figure}

\section{Software stanice}

Na staničním počítači je nainstalovaný operační systém linux Lubuntu 14.04 LTS včetně grafického rozhraní xfce.  Grafické rozhraní je na stanici využíváno pro zobrazování stavových informací. (živý waterfall, stav GPS, stav uploadu dat). S datovým tokem z přijímače je zacházeno jako s širokopásmým audio signálem. Proto je pro rozvod  datového toku použit audiosystém JACK. Příklad rozvodu signálu je na obrázku \ref{qjackctl}.

\begin{figure}[htbp]
\centering
\includegraphics [width=120mm] {./img/JACK_audio.png} 
\caption{Rozvod audio signálu v systému JACK.}
\label{qjackctl}
\end{figure}

Jack audio server navíc umožňuje přenášet stream přes lokální síť ethernet a aktuální tok spektra z přijímače je tak možné sledovat i na jiných počítačích viz obr. \ref{Pysdr}.

\begin{figure}[htbp]
\centering
\includegraphics [width=120mm] {./img/meteor_pysdr.jpg} 
\caption{Vzdálené zobrazení toku spektra v programu PySDR. S meteorem označeným MIDI packetem.}
\label{Pysdr}
\end{figure}

\subsection{Záznam dat}

Data jsou na stanici zaznamenávána programem radio-observer, který běží jako klient systému jack, počítá FFT spektra a detekuje meteory podle intenzity signálu v oblasti, kde je předpokládán signál od vysílače. Pokud v této oblasti dojde ke zvýšení intenzity signálu o prahovou hodnotu nad šumem tak je detekován meteor.  Při detekci meteoru vznikne několik záznamů, jednak náhled na frekvenční oblast spektra, kde meteor byl detekován. Potom záznam surových vzorků z přijímače a také záznam do metadatového souboru s parametry signálu detekovaného meteoru (délka odrazu, intenzita signálu, atd.). Datové výstupy jsou ve formátu FITS s kompresí RICE, buď jako navzorkované signály a nebo předpočítané spektrogramy. Všechna naměřená data jsou veřejně dostupná na datovém serveru. 

\subsection{Odesílání záznamů}

Naměřená data jsou ještě na stanici setříděna do stromové adresářové struktury podle času. A následně uploadována přes rsync na hlavní datový server space.astro.cz.

\subsection{Správa stanice}

Na stanici běží ještě další pomocné aplikace, jako sshd server, pro vzdálený přístup a skripty pro ladění a kontrolu lokálního oscilátoru.


\begin{thebibliography}{99}
\bibitem{GRAVES_book}{GRAVEs source book} 
\href{http://www.fas.org/spp/military/program/track/graves.pdf}{http://www.fas.org/spp/military/program/track/graves.pdf}

\bibitem{LNA01A_wiki}{Nízkošumový zesilovač LNA01A} 
\href{http://wiki.mlab.cz/doku.php?id=cs:lna}{http://wiki.mlab.cz/doku.php?id=cs:lna}

\bibitem{SDRX01B_pdf}{Softwarově definovaný přijímač SDRX01B} 
\href{http://www.mlab.cz/Designs/HAM\%20Constructions/SDRX01B/DOC/SDRX01B.cs.pdf}{http://www.mlab.cz/Designs/HAM\%20Constructions/SDRX01B/DOC/SDRX01B.cs.pdf}

\bibitem{LNA01A_wiki}{Síť Bolidozor} 
\href{http://wiki.bolidozor.cz/}{http://wiki.bolidozor.cz/}

\bibitem{meteor_distance}{Meteor distance parameters} 
\href{http://www.amsmeteors.org/richardson/distance.html}{http://www.amsmeteors.org/richardson/distance.html}


\end{thebibliography}
\end{document}